\chapter{Shortest Path Problems}
\section{Introduction}
% Something about the importance of this problem family
\section{All pairs shortest path}
Here we see the power of exchanging the semiring for the first time. In this section we will use the tropical semiring $(\R \cup \{\infty\}, \min, +)$ and its induced matrix operations. But first we have to state the problem:
\begin{problem}[All pairs shortest path] 
    Given a directed weighted graph $G = (V, E)$ with $|V| = n$ and $|E| = m$ and its adjacency matrix $A$. Compute the length of the shortest path between every pair of nodes.
\end{problem}

\begin{lemma}
    Let $A \in \R^{n \times n}$ be an adjacency matrix. Then $(A^k)_{ij}$ holds the length of the shortest path with $k+1$ nodes between $v_i$ and $v_j$ for all $k \in \N$. 
\end{lemma}
\begin{proof}
    Let $D_k \in \R^{n \times n}$ ba a matrix such that $(D_k)_{ij}$ holds the length of the shortest path with $k+1$ nodes between $v_i$ and $v_j$. Thus we need to show that $A^k = D_k$ for all $k \in \N$ which we will do by induction. Namely, we need to show that (I) $D_0 = \imat_n$ and (II) $D_{k+1} = D_k \odot A$
    \begin{enumerate}
        \item[(I)] ...
        \item[(II)] ...
    \end{enumerate}
\end{proof}

\begin{lemma}
    Let $A \in \R^{n \times n}$ be an adjacency matrix of a directed weighted graph $G$. Then $G$ has no negative cycle $\Leftrightarrow$ 
    $$\sum_{k=0}^{n-1+q} A^k = \sum_{k=0}^{n-1} A^k \quad\forall q \in \N$$
\end{lemma}

\begin{corollary}
    A direct conclusion of the last lemma is that
    $$A^* = \sum_{k=0}^{\infty}A^k = \sum_{k=0}^{n-1}A^k$$
\end{corollary}

And $A^*$ solves our problem.

Our goal was more ambitious than just solving the problem. We wanted to state the solution in the language of Einsums. For that we need to go up one dimension. Let $R_{ijk} := A^{n-1}_{ij}$ be a tensor. Now we can rewrite $A^*$ as follows
$$A^*_{ij} = \sum_{k=0}^{n-1} A^k_{ij} = \sum_{k=1}^{n}A^{k-1}_{ij} = \sum_{k=1}^{n}R_{ijk}$$
This expression is allready in the necessary shape. The last step is to compute $R_{ijk}$. For that we need to define even more tensors: 
$$A^{(l)}_{ijk} := 
\begin{cases}
    A_{ij} & \textrm{if}\quad l \leq k\\
    \imat_{ij} & \textrm{else}   
\end{cases}
$$
As shown in Fig. \ref*{fig:shortest_path_tensor} we just need to multiply all matricies at each level together and than add the result up.

\begin{figure}[h]
    \includegraphics[width=\linewidth]{shortest_path_tensor.png}
    \caption{Visualized the shortest path tensors}
    \label{fig:shortest_path_tensor}
\end{figure}

The resulting expression comes out to be
$$A^*_{ij} = \sum_{k, t_1, \dots, t_{n-1} \in [n]} A^{(1)}_{it_1k}A^{(2)}_{t_1t_2k}A^{(3)}_{t_2t_3k}\dots A^{(n-1)}_{t_{n-2}t_{n-1}k}A^{(n)}_{t_{n-1}jk}$$
Which correspond to the Einsum string
$$it_1k, t_1t_2k, t_2t_3k, \cdots, t_{n-2}t_{n-1}k, t_{n-1}jk \to ij$$

\section{All pairs longest path}
The computation is exactly the same as in the All pairs shortest path problem. The only difference is the semiring in use. Here we need $(\R \cup \{-\infty\}, \max, +)$ and there must not exist a positive cycle. Then just compute $A^*$ and the entry $A^*_{ij}$ holds the longest path from $v_i$ to $v_j$. [Small note on convergency of $A^*$]

\section{Minimum weight spanning tree}
Here we need the observation that the edge $(v, u)$ is not included in the MST iff its weight is larger than the maximum weight of any path between $v$ and $u$. [Add proof] This we can model, by defining the weight of a path as the maximum of the weights of its edges. Then we compute the all pairs shortest path problem, but now we have exchanged the $+$-operation by the $\max$-operation, which means we have to use the $(\R \cup\{\infty\}, \min, \max)$ semiring. After computing $A^*$ we include the edge $(v_i, v_j)$ in the minimum weight spanning tree iff $A_{ij} \leq A^*_{ij}$.
[Diskuss conergency of $A^*$]

\section{Further Problems}
\begin{itemize}
    \item Marko Chaines. Using the semiring $([0, 1], +, \cdot)$ $A^k_{ij}$ holds the propability that an agent reaches $v_j$ starting at $v_i$ after $k$ steps if the weights of the edges symbolze the properbility that an agent moves along this edge.
    \item Reachability. In an undirected unweighted graph, we ask the question whether a vertex $v_j$ is reachable starting at vertex $v_i$. For that we could just compute the shortest path and see whether $A^*_{ij}$ is infinite or not. But we can achieve the same with less information by just using the $(\{0, 1\}, \lor, \land)$ semiring and check whether $A^*_{ij} = 1$.
\end{itemize}