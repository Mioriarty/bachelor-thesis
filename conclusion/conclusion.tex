\chapter{Conclusion}
In this thesis, we revisited and extended the approach proposed by \cite{algebraic_statistics} for solving integer linear programming (ILP) instances by transforming them into instances of the shortest path problem on a directed and weighted graph. We focused on a broader class of ILP instances, allowing for varying column sums in the constraint matrix $A$.

Through our research, we successfully extended the proposed algorithm to handle matrices with positive column sums, thereby introducing a new perspective on ILP solving. While our algorithm may not outperform existing approaches in terms of speed, it offers valuable insights and alternative strategies for tackling ILP problems.

The implications of our findings lie in the exploration of different viewpoints and techniques for ILP optimization. By extending the applicability of the algorithm to more general matrices, we open up possibilities for further research and development in this area.

Moving forward, we recommend further exploration into the extension of our approach to handle general whole number matrices. This could involve refining the algorithm and investigating additional strategies to improve its performance and efficiency.

Reflecting on the research process, I am grateful for the guidance and support provided by my advisor throughout this journey. Their encouragement allowed me to freely explore my mathematical intuition and pursue this research with passion.

% In closing, I would like to express my gratitude to all those who contributed to this thesis, and thank you for the opportunity to delve into this fascinating area of study.